\section*{Introduction}

\begin{itemize}
    \item Tuesday 10:15 - 12:00
    \item Thursday: 08:25 - 10:00
    \item Orga infos and literature on ecampus
\end{itemize}

\section{Partial differential equations}

\subsection{Laplace equation}

\underline{\textbf{Problem:}} How to model soap membrane spanned by a wire sling?

\underline{\textbf{Notation:}} 

\begin{itemize}
    \item $\Omega\subset \R^2$ a bounded domain (open and connected set)
    \item $\Gamma = \partial \Omega$
    \item $g: \Gamma\to \R$ describing the wire sling
    \item $u:\Omega:\to\R$ describing the soap membrane
\end{itemize}

% graphic

\underline{\textbf{Question:}} Given $\Omega$ and $g$, how can we characterize the soap membrane?

$u$ has minimal surface area.

\[\min_u \int_{u(\Omega)} 1 d\sigma=\int_\Omega \left\Vert \overrightarrow{u_x}\times \overrightarrow{u_y} \right\Vert_2 dxdy=\left\Vert \begin{pmatrix}
    1\\0\\u_x(x,y)
\end{pmatrix} \times \begin{pmatrix}
    0\\1\\u_y(x,y)
\end{pmatrix}\right\Vert_2=\left\Vert \begin{pmatrix}
    -u_x(x,y)\\u_y(x,y)\\1
\end{pmatrix} \right\Vert_2=\sqrt{1+u_x(x,y)^2+u_y(x,y)^2}\]

\underline{\textbf{Observation:}} $\sqrt{1+z}=1+z+O(z^2),z\to 0$.

$\implies $ Alternate minimization problem: 
\[\min_{u} \underbrace{\frac{1}{2}\int_\Omega \left(u_x(x,y)^2+u_y(x,y)^2 \right) dx dy}_{F(u)}=\min_u F(u)\]


\underline{\textbf{Assume: }} We have a minimizer $u\in C^2(\Omega)\cap C(\overline{\Omega})$ with $u\vert_{\Gamma}=g$ % ?

For $v\in C^1(\Omega)\cap C(\overline{\Omega})$ with $v\vert_\Gamma=0$, we obtain 
\begin{align}
    0&=\lim_{\epsilon\to 0}\frac{F(u+\epsilon)-F(u)}{\epsilon}\\
    &=\lim_{\epsilon\to 0} \frac{1}{2\epsilon}\int_\Omega\left((u_x+\epsilon v_x)^2+ (u_y+ \epsilon v_y)^2- (u_x^2+u_y^2) \right)dydx = (\star)\\
    &= \lim_{\epsilon\to 0} \frac{1}{2 \epsilon}\int_{\epsilon\to 0}\int_\Omega (2 \epsilon u_x v_x+\epsilon^2 v_x^2+2 \epsilon u_y+v_y+ \epsilon^2v_y^2)dxdy\\
    &=\lim_{\epsilon\to 0}\frac{1}{2}\int_\Omega(2u_xv_x+\epsilon v^2+ 2u_yv_y+\epsilon v_y^2)dxdy\\
    &=\int_\Omega(u_xv_x+u_yv_y)dxdy\\
    &=\int_\Omega \langle  \nabla u, \nabla v \rangle dxdy% (1.1)
\end{align} 

\underline{\textbf{Observation:}} A similar term as (1.1) also appears in the Gaus theorem, i.e.

\[\int_\Omega\text{div} \overrightarrow{f}\overrightarrow{x}=\int_\Gamma \langle \overrightarrow{f},\overrightarrow{n} \rangle d\sigma \]

where $\overrightarrow{n}$ is the outward pointing normal to $\Gamma$, and $f:\Omega\to\R^3$. If $f=\nabla(u)v$ we obtain

\[\]


\begin{align*}
    \int_\Omega \text{div}(\nabla u)v dx+ \int_\Omega \langle \nabla u,\nabla v \rangle dx\\
    =\int_\Omega \text{div} f dx\\
    =\int_\Gamma \langle f, n\rangle d\sigma\\
    =\int_\Gamma\frac{\partial u}{\partial n} \underbrace{v}_{=0} d\sigma =0
\end{align*}

Summarizing, $u$ needs to satisfy 

\[\int_\Omega  \underbrace{\text{div}(\nabla u)}_{=\Delta u}vdx=0\]
for all $v\in C^1(\Omega)\cap C(\overline{\Omega})$

By the fundamental lemma of calculus of variations, u must satisfy
\[\begin{cases}
    \Delta u (x)=0 & x\in\Omega\\
    u(x)=g(x) & x\in \Gamma
\end{cases}\]

This equation is called the \underline{\textbf{Laplace equation}} or a \underline{\textbf{potential equation}}.



\subsection{Heat equation}

\underline{\textbf{Problem:}} How to model temperature in a fixed volume over time?

\underline{\textbf{Notation:}} 
\begin{itemize}
    \item $\Omega\subset \R^d$ a bounded domain, $d\in \N_+$
    \item $\Gamma = \partial \Omega$
    \item $u: \R_{>0}\times \Omega \to \R$
    \item $u(0,x)=u_0(x),x\in \Omega$
    \item $u(t,x)=g(t,x)$ $(t,x)\in \R_{>0}\times \Omega$.
    \item $f(t,x)$ heat source in $\Omega$ $(t,x)\in \R_{>0}\times \Omega$.
\end{itemize}

\underline{\textbf{Question:}} Given $\Omega,u_0,f,g$ how can we characterize $u$?

First law of thermodynamics:

For any given $V\subset \Omega$ it must hold
\[\underbrace{\int_V \frac{\partial}{\partial}u(t,x)dx}_{\text{Change of temperature in }V}=\underbrace{-\int_{\partial V} \langle q(t,x),n \rangle d\sigma}_{heat flow through }V+\int_V f(t,x)dx\]
with the material law
\[q(t,x)=c(x\nabla u(t,x)), c(x)\geq c_0>0\]

\begin{align*}
    \implies \int_{\partial V} \langle q(t,x),u \rangle d\sigma  &= \int_V\text{div}(q(t,x))dx\\
    &=\int_V\text{div}(c(x)\nabla u(t,x))dx\\
    \implies \int_v \frac{\partial}{\partial t} U(t,x)-\text{div}_x((c(x)\nabla u(t,x)))dx &=\int_V f(t,x)dx
\end{align*}

By a variation of the fundamental lemma of calculus of variations we obtain the \underline{\textbf{heat equation}}
 \[\begin{cases}\frac{\partial}{\partial t} u(t,x)-\text{div}(c(x)\nabla u(t,x))=f(t,x)& (x,t\in\R_{>0}\times \Omega)\\
    u(0,x)=u_0(x) & x\in \Omega\\
    u(t,x)=g(t,x) & (t,x) \in \R_{>0}\times \Gamma
\end{cases}
 \]

For $c(x)=1$ and $f,g$ time independent, the temperature will tend towards an equilibrium for $t\to\infty$. 
Then $\frac{\partial u}{\partial {t}}=0$ and we obtain the poisson equation
\[\begin{cases}
    -\delta u(x)=f(x) & x\in \Omega \\
    u(x)=g(x)&x\in\Gamma
\end{cases}\]


\subsection{Wave equation}

\underline{\textbf{Problem:}} How can we model waves in an ideal gas?

\underline{\textbf{Notation:}}
\begin{itemize}
    \item $\Omega\subset \R^d$ a bounded domain, $d\in \N_{>0}$
    \item $\Gamma=\partial \Omega$
    \item velocity $v:\R_{>0}\times \Omega\to \R^d$ of particles
    \item density $\rho=\rho_0+\rho_1:\R_{>0}\times \Omega \to \R$, $\rho_0$ is constant with $|\rho_1|<<\rho_0$
    \item pressure $p:\R_{>0}\times \Omega\to \R$ of gas
    \item $p(0,x)=p_0(x), t=0$
    \item $g(t,x)$ pressure at boundary $\Gamma$, at $t>0$
\end{itemize}

\underline{\textbf{Question:}} Given $\Omega,v,\rho,p_0,g$ can we characterize $p$?

\underline{\textbf{1. Contiuity Equation:}} (mass conservation) in any $V\subset \R$.

\begin{align*}
    \underbrace{\int_V\frac{\partial}{\partial t}\rho(t,x)dx}_{\text{Change of mass}}= - \underbrace{\int_{\partial V} \rho(t,x) \langle v(t,x),n \rangle d\sigma}_{\text{flux through }\partial V}\\
    =\kappa - \rho_0 \int_{\partial V} \langle v(t,x),n \rangle d\sigma \\ 
    =-\rho_0 \int_V \text{div}_x(v(t,x))dx
\end{align*}

As before $\implies$ 

\[\frac{\partial}{\partial t}\rho(t,x)\approx -\rho_0 \text{div}_x(v(t,x))\]

\underline{\textbf{2. Newton's law:}} 
\begin{align*}
    -\nabla_x p(t,x)=\rho(t,x)\frac{\partial}{\partial}v(t,x)\approx \rho_0\frac{\partial}{\partial t}v(t,x)
\end{align*} 

\underline{\textbf{3. Equation of state: }} 
\[p(t,x)=c^2\rho(t,x)\]

Combining these 3 laws:

\begin{align*}
    \frac{\partial^2}{\partial t^2}p(t,x)&=c^2\frac{\partial^2}{\partial t^2}\rho(t,x) \\
    &= -c^2 \frac{\partial}{\partial t}\text{div}:x(\rho_0 v(t,x)) = (\star) \\
    &=-c^2 \text{div}_x(\rho_0 \frac{\partial}{\partial t} v(t,x))\\
    &=c^2\text{div}_x(\nabla p(t,x))\\
    &= c^2\Delta p(t,x)
\end{align*}


This yields the \underline{\textbf{wave equation}}

\[\begin{cases}
    \frac{\partial^2}{\partial t^2}p(t,x-\Delta p(t,x))=0 & (t,x\in\R_{>0}\times \Omega)\\
    p(t,x)=g(t,x) & (t,x)\in\R_{>0}\times \Gamma\\
    p(0,x)=p_0(x) & x\in \Omega
\end{cases}\]




