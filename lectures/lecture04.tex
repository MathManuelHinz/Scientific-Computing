

\underline{\textbf{Idea:}} For $x\in\Omega$, we have 
\[\Delta u(x)=-\sum_{i=1}^d (\partial_i^{+h}\partial_i^{-h}u)(x)+O(h^2)\]
\underline{\textbf{Idea:}} Introduce a equally spaced grid on $\Omega$:
%Pic
Discrete Domain $\Omega_h=\{x\in \Omega: x=hk,k\in\Z^d\}$.

Discrete Boundary $\Gamma_h=\{x\in\Gamma:\exists 1 \leq i \leq d: x_i=hk_i,k_i\in \Z\}$.

Set $\overline{\Omega}_h=\Omega_h\cup\Gamma_h$, $\overline{\Omega}_h\setminus \Gamma_h$ interior points.

\underline{\textbf{Observation:}} If $\Omega$ is a disjoint union of equal cubes, then
we can modify (2.1) to \[\begin{cases}
    \Delta_h u_h(x)=f(x) & \in\overline{\Omega}_h\setminus \Gamma_h\\
    u_h(x)=g(x) & x\in \Gamma_h
\end{cases}\]

This corresponds to a system of linear equations!

\begin{example}
    $\Omega=(0,1)^2,n\in\N,h=\frac{1}{n},y_{ij}h(i,j),i,j=0,\dots,n$

    Assume: $g=0$, Abbreviate $u_{ij}\coloneqq u(x_{ij})$, then 
\begin{align*}
    \Delta_hu_h(x_{ij})&=\frac{4u_{ij}-u_{i-1j}-u_{i+1j}-u_{ij-1}-u_{ij+1}}{h^2}\\
    &=\frac{1}{h^2}\begin{bmatrix}
        &-1 &\\
        -1 & 4 & -1\\
        & -1 &
    \end{bmatrix}u(x_{ij})
\end{align*}

This is called the \underline{\textbf{5-point finite difference stencil}}. Set $f_{ij}\coloneqq f(x_{ij})$.

\begin{align*}
    \implies \frac{4u_{ij}-u_{i-1j}-u_{i+1j}-u_{ij-1}-u_{ij+1}}{h^2} = f_{ij}& i=1,\dots, n-1\\
    u_{ij}=0 & i=\{0,1\}, j\in \{0,1\}.
\end{align*}

We cna write this in matrix form as 
\begin{align*}
    \frac{1}{h^2}\begin{bmatrix}
        A  & -I &  &  &&\\
        -I & A & -I &  & & \\
        & \ddots & \ddots & \ddots &&\\
        &&\ddots &\ddots & -I\\
        &&&-I&A
    \end{bmatrix}\begin{bmatrix}
        u_1\\
        \vdots\\
        \vdots\\
        u_{n-1}
    \end{bmatrix}=\begin{bmatrix}
        f_1\\
        \vdots\\
        \vdots\\
        f_{n-1}
    \end{bmatrix}
\end{align*}

where 

\begin{align*}
    A=\begin{bmatrix}
        4 & -1 & & & & \\
        -1 &\ddots & \ddots& & &\\
        & \ddots &\ddots & \ddots& &\\
        && \ddots & \ddots & \ddots \\
        & &&\ddots & \ddots & -1\\
        &&&&-1&4
    \end{bmatrix}
    u_i=\begin{bmatrix}
        u_{i,1}\\
        \vdots\\
        u_{i,n-1}
    \end{bmatrix}
    f_i=\begin{bmatrix}
        f_{i,1}\\
        \vdots\\
        f_{i,n-1}
    \end{bmatrix}
\end{align*}

\end{example}


\begin{remark}
    The same approach can be applied to more partial differential operators, but a few technicalities need to be considered.
\end{remark}

\begin{remark}
    More general domains (non-cuboids) can be dealt with by modifying the stencils close to the boundary. However in general, this will only lead to $O(h)$ approximations, rather than $O(h^2)$    
\end{remark}

\underline{\textbf{Question:}} Are the systems of linear equations uniquely solvable?

\subsection{The discrete maximum principle}

\underline{\textbf{Observation:}} Applying a finite difference stencil amounts to computing a weighted average of functions values.
The value of the stencil can not be larger as the maximum of the values where it is applied to.

\begin{lemma}[Star lemma]
    Let $k>0$ and consider numbers $\alpha_0,\dots,\alpha_k$ with $\alpha_1,\dots,\alpha_k<0$ and $p_0,\dots,p_k$ such that
    \[\sum_{l=0}^k \alpha_l\geq 0, \sum_{l=0}^k \alpha_lp_l\leq 0.\]

    Assume that $p_0\geq 0$ or $\sum_{l=0}^k\alpha_l=0$, then if $p_0\geq \max_{1\geq l\geq k} p_l$, it holds that 
    \[p_0=p_1=\dots=p_k\]
\end{lemma}

\begin{proof}
\begin{align*}
    0&\geq \sum_{k=0}^l \alpha_l p_l-p_0\sum_{l=0}^k\alpha_l\\
    =\sum_{l=0}^k \alpha_l(p_l-p_0)=\sum_{l=1}^k \underbrace{\underbrace{\alpha_l}_{< 0} \underbrace{(p_l-p_0)}_{\geq 0}}_{\geq 0} \geq 0
\end{align*}
which implies the assertion.    
\end{proof}

\begin{theorem}[Discrete maximum principle]

    Let $u_h$ be the solution to $\Delta_h u_h=f$, with $f\leq 0$ and assume that $\Omega_h$ is (discretely) connected. Then it holds
    \[\max_{x\in \overline{\Omega}_h \setminus \Gamma_h} u_h(x)\leq \max_{x\in\Gamma_h} u_h(x).\]

\end{theorem}

\begin{proof}
    Assume that the maximum is attained at $y\in \overline{\Omega}_h \setminus \Gamma_h$, set $p_0=u_h(y)$. Identify $p_1,\dots,p_k$ with 
    the values of $u_h$ on the neighboring grid cells, and $\alpha_0,\dots,\alpha_k$ with the weights of the stencil.

    \begin{align*}
        0\geq f(y) = (\Delta_h u_h)(y) = \sum_{l=0}^k \alpha_l p_l.\\
        \stackrel{\text{Star lemma}}{\implies} p_0=\dots = p_k
    \end{align*}
    i.e., $u_h(y)$ at points in eastern, western, southern and northern direction is equal to $u_h$.

    Proceed iteratively by marching to the boundary.
\end{proof}

All implications of the continuous case transfer to the discrete case. Most importantly:
\begin{corollary}
    Under the assumptions of (last theorem), the solution of (2.2) is unique.
\end{corollary}




